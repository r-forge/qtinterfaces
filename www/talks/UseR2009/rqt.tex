\documentclass[compress]{beamer}
% \documentclass[t,compress]{beamer}
% \usepackage{multimedia}



% Although R is primarily a command-line tool, it provides facilities
% for interfacing with various GUI toolkits. Qt is a powerful, robust,
% and cross-platform GUI toolkit licensed under the GPL. Full use of
% Qt from R requires bindings to Qt, which are not yet
% available. However, it is relatively simple to write custom tools
% that use Qt for specific purposes. In this talk, I will give some
% examples of such tools, including a R graphics device implemented
% using Qt, an alternative R help browser with full-text search
% capabilities, a simple data import wizard, and a basic but
% functional high-level graphics system not unlike lattice, that is
% completely independent of the R graphics engine.

\usepackage{times}
\usepackage[T1]{fontenc}

\title{Using Qt for GUI Tasks and Graphics in R}

\author{Deepayan Sarkar, Michael Lawrence}
\institute[FHCRC] % (optional, but mostly needed)
{
  Computational Biology\\
  Fred Hutchinson Cancer Research Center
}
\date{July 10, 2009}

%% suppress navigation symbols
\setbeamertemplate{navigation symbols}{}

%% Logo

\pgfdeclareimage[height=0.6cm]{university-logo}{hutchlogo}
\logo{\pgfuseimage{university-logo}}

\usepackage{Sweave}
\usepackage{SweaveBeamer}
\usepackage{alltt}
% \usepackage{amsbsy}
\usepackage{amsmath}
\usepackage{amsfonts}
\usepackage{amssymb}
\usepackage{amsthm}
\usepackage{bm}
\usepackage{graphicx}
\usepackage{url}
\usepackage{hyperref}
\setkeys{Gin}{width=\textwidth}

%\newcommand{\fixme}[1]{\emph{\small \textbf{#1}}}
%% \newcommand{\R}{\textsf{R}}
%% \renewcommand{\emph}[1]{\alert{\textit{#1}}}

\newcommand{\R}{\Emph{R}\xspace}
\newcommand{\SPLUS}{\Emph{S-PLUS}}
\newcommand{\Emph}[1]{\emph{\color{Scode}#1}}
\newcommand{\trans}{\ensuremath{^\prime}}


\newcommand{\lattice}{\code{lattice}}
\newcommand{\Rpackage}[1]{\code{#1}}
\newcommand{\Rfunction}[1]{\code{#1()}}
\newcommand{\class}[1]{\textit{``#1''}}



%% \newcommand{\code}[1]{\texttt{#1}}

% \setbeamercolor{title}{fg=blue!50!black}


%% \SweaveOpts{width=9,height=6.5,keep.source=TRUE}
%% \SweaveOpts{prefix=TRUE,prefix.string=figs/ShortReadFig,include=TRUE,pdf=TRUE,eps=FALSE}


\mode<presentation>
{
  % \usetheme{default}
  \usetheme[width=0cm]{Hannover}
  %\useoutertheme{split}
  \setbeamercovered{transparent}
}

\begin{document}
\begin{frame}
  \titlepage
\end{frame}


%% \section{Introduction}


% \begin{frame}
%   \frametitle{\Rpackage{ShortRead} Workflow}
%   \begin{itemize}
%   \item Read in data (e.g., raw data, aligned reads)
%   \item Quality assessment
%   \item Analysis (e.g., find transcription factor binding sites)
%   \item Interpretation and further analysis
%   \end{itemize}
% \end{frame}

% \begin{frame}
%   \frametitle{\Rpackage{ShortRead} Workflow}
%   \begin{itemize}
%   \item We will focus on simple examples of 
%     \begin{itemize}
%     \item data input
%     \item quality assessment
%     \item finding TF binding sites
%     \end{itemize}
%   \end{itemize}
% \end{frame}

\begin{frame}
  \frametitle{R and Qt}
  \begin{itemize}
  \item \R: primarily command-line (Read-eval-print loop)
    \begin{itemize}
    \item Allows GUI front-ends to embed \R
    \item Facilitates bindings to external libraries; e.g. 
      \begin{itemize}
      \item libxml, \Emph{tcltk}, \Emph{RGtk2}
      \end{itemize}
    \end{itemize}
  \item Qt: powerful C++ Application and UI framework
    \begin{itemize}
    \item Mature, cross-platform, high-performance
    \item Latest release (Qt 4.5) is GPL/LGPL on all platforms
    \item Good candidate for binding to R
    \end{itemize}
  \end{itemize}
\end{frame}


\begin{frame}
  \frametitle{Focus areas}
  \begin{itemize}
  \item Improve access to \R\ documentation
    \begin{itemize}
    \item Difficult to search (Java search engine?)
    \item Inaccessible when R is running
    \item Multiple matches not handled gracefully
    \item S4 method documentation hard to find
    \end{itemize}
  \item Enable Qt GUI programming
    \begin{itemize}
    \item Beneficial for many problems \\ 
      ~~~~e.g., domain-specific GUI, graphical debugger
    \item Should be accessible to R programmers
    \end{itemize}
  \item Graphics (both static and dynamic)
    \begin{itemize}
    \item R graphics powerful for static graphics, but slow (grid!)
    \item Not designed for dynamic manipulation 
    \end{itemize}
  \end{itemize}
\end{frame}




% \begin{frame}
%   \frametitle{Focus areas}
%   \begin{itemize}
%   \item Use \Emph{Qt Assistant} as an interface to \R\ documentation
%   \item Enable Qt GUI programming
%     \begin{itemize}
%     \item Allow as much R-level control as practical
%       \begin{itemize}
%       \item use Qt dynamic features as much as possible 
%       \item signals connected to R functions
%       \item trade-off between utility and amount of manual work 
%       \end{itemize}
%     \item More powerful extensions by writing custom C++ classes 
%     \end{itemize}
%   \item Use Qt's Graphics/View framework and layout capabilities for
%     advanced graphics (both static and dynamic)
%   \end{itemize}
% \end{frame}


%% major points: philosophically, immediate audience is R programmers,
%% not end-users.  Many intermediate developers come up with cool
%% applications, but directly interfacing C/C++ etc. is a big barrier.
%% We hope to lower that barrier as much as possible, while exposing
%% the extensive power of Qt to the R programmer. Target the
%% distinguishing qualities of the S language: 





\begin{frame}
  \frametitle{R and Qt}
  \begin{itemize}
  \item \Emph{RKWard}: Qt/KDE based front-end embedding \R
  \item \Emph{RQt}
    \begin{itemize}
    \item Proof-of-concept Omegahat project (Duncan Temple Lang)
    \item Long-term focus on automatic generation of bindings
      \end{itemize}
    \item \Emph{qtinterfaces}
      \begin{itemize}
      \item Recently registered R-forge project
      \item Initial developers: \\
        ~~~Deepayan Sarkar, Michael Lawrence, Hadley Wickham
      \item Goal:\\
        %\begin{center}
        \Emph{
          ~~~Create a coherent collection of R packages that provide \\
          ~~~an interface to the Qt application and UI framework, \\
          ~~~with a focus on enabling GUI development and \\
          ~~~advanced graphics.}
        %\end{center}
      \end{itemize}
    \end{itemize}
\end{frame}


\begin{frame}
  \frametitle{Some Demos}
  \begin{itemize}
  \item Using Qt Assistant to view R documentation
  \item ``Object viewer'' with applications
  \item An \R\ graphics device implemented using Qt's Graphics/View framework
  \item Mosaiq: a high-level graphics system not unlike lattice, but
    completely independent of the R graphics engine.
  \item RGtk2 driving a dynamic Qt view 
  \end{itemize}
\end{frame}


\begin{frame}
  \frametitle{Qt Assistant}
  \begin{itemize}
  \item Standalone Qt library documentation viewer
  \item Can also be used as viewer for third-party static HTML files
    \begin{itemize}
    \item Requires preprocessing (add keywords, register, compile)
    \item Once processed, much more powerful help interface than anything R has
    \end{itemize}
  \item \Emph{But}, R documentation is dynamic!
  \item Incompatible models, but we can get snapshots, almost as useful
  \end{itemize}
\end{frame}



%  o R graphics device implemented using Qt
 
%  o an alternative R help browser with full-text search capabilities
 
%  o a simple data import wizard, 
 
%  o basic but functional high-level graphics system not unlike lattice,
%    that is completely independent of the R graphics engine.


\begin{frame}
  \frametitle{Qt GUI programming}
  \begin{itemize}
  \item Creation and memory management of widgets and other Qt objects
    (as external pointers)
  \item Manipulate these objects using R functions
  \item Connect Qt signals to arbitrary R functions
  \item Examples:
    \begin{itemize}
    \item ``Object viewer'' with applications
    \item Data import wizard
      %% R graphics device based on a scene graph
    \end{itemize}
  \end{itemize}
\end{frame}


\begin{frame}
  \frametitle{Graphics}
  \begin{itemize}
  \item Based on the Scene Graph model; supports layers and layouts
  \item Fast implementation of low-level drawing
    \begin{itemize}
    \item Custom OpenGL layer
    \item Default Qt abstraction (output to PDF, SVG, pixmap, etc.)
    \end{itemize}
  \item Callbacks for painting and events (everything happens in R)
  \item Examples:
    \begin{itemize}
    \item R graphics device based on a scene graph
    \item Mosaiq: a high-level graphics system similar to lattice
    \end{itemize}
  \end{itemize}
\end{frame}


%% Discussion points: some manual binding, some automatic (using Qt's
%% dynamic invocation and dynamic callbacks using Qt's properties
%% meta-data features )

%% GO graph with links

\begin{frame}
  \frametitle{An animated tour}
  \begin{itemize}
  \item Contributed by Hadley and Bei Bei 
  \item An RGtk2 GUI control driving a Qt canvas  
  \end{itemize}
\end{frame}

\end{document}


\begin{frame}
  \frametitle{}
  \begin{itemize}
  \item 
  \end{itemize}
\end{frame}

\begin{frame}
  \frametitle{}
  \begin{itemize}
  \item 
  \end{itemize}
\end{frame}

\begin{frame}
  \frametitle{}
  \begin{itemize}
  \item 
  \end{itemize}
\end{frame}

\begin{frame}
  \frametitle{}
  \begin{itemize}
  \item 
  \end{itemize}
\end{frame}

\begin{frame}
  \frametitle{}
  \begin{itemize}
  \item 
  \end{itemize}
\end{frame}

