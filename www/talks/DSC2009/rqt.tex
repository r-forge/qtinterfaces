\documentclass[compress]{beamer}
% \documentclass[t,compress]{beamer}
% \usepackage{multimedia}


% R is well known for its powerful graphics capabilities; however,
% native R graphics is fundamentally limited by its ink-on-paper
% model.  Grid graphics provides enormous flexibility within this
% model, but at the cost of speed.  Part of the flexibility of grid
% arises from notions borrowed from the world of graphical toolkits,
% such as layouts and self-sizing graphical objects.  It is a natural
% next step to explore the direct use of graphical toolkits for
% graphics and other similar tasks in R.  Qt is a powerful, robust,
% and cross-platform GUI toolkit licensed under the GPL.  In addition
% to features expected in a GUI toolkit, Qt has a flexible painting
% system that makes implementation of statistical graphics relatively
% easy.  In this talk, I will describe a low-level interface that
% provides R-level tools to manipulate graphical primitives and
% layouts in Qt, and demonstrate a basic but functional high-level
% static graphics system, not unlike lattice, that is implemented
% using these low-level tools.  I will also outline other potential
% applications of an R-Qt interface.


\usepackage{times}
\usepackage[T1]{fontenc}

\title{Using Qt for GUI Tasks and Graphics in R}

\author{Deepayan Sarkar, Michael Lawrence}
\institute[FHCRC] % (optional, but mostly needed)
{
  Computational Biology\\
  Fred Hutchinson Cancer Research Center
}
\date{July 10, 2009}

%% suppress navigation symbols
\setbeamertemplate{navigation symbols}{}

%% Logo

\pgfdeclareimage[height=0.6cm]{university-logo}{hutchlogo}
\logo{\pgfuseimage{university-logo}}

\usepackage{Sweave}
\usepackage{SweaveBeamer}
\usepackage{alltt}
% \usepackage{amsbsy}
\usepackage{amsmath}
\usepackage{amsfonts}
\usepackage{amssymb}
\usepackage{amsthm}
\usepackage{bm}
\usepackage{graphicx}
\usepackage{url}
\usepackage{hyperref}
\setkeys{Gin}{width=\textwidth}

%\newcommand{\fixme}[1]{\emph{\small \textbf{#1}}}
%% \newcommand{\R}{\textsf{R}}
%% \renewcommand{\emph}[1]{\alert{\textit{#1}}}

\newcommand{\R}{\Emph{R}\xspace}
\newcommand{\SPLUS}{\Emph{S-PLUS}}
\newcommand{\Emph}[1]{\emph{\color{Scode}#1}}
\newcommand{\trans}{\ensuremath{^\prime}}


\newcommand{\lattice}{\code{lattice}}
\newcommand{\Rpackage}[1]{\code{#1}}
\newcommand{\Rfunction}[1]{\code{#1()}}
\newcommand{\class}[1]{\textit{``#1''}}



%% \newcommand{\code}[1]{\texttt{#1}}

% \setbeamercolor{title}{fg=blue!50!black}


%% \SweaveOpts{width=9,height=6.5,keep.source=TRUE}
%% \SweaveOpts{prefix=TRUE,prefix.string=figs/ShortReadFig,include=TRUE,pdf=TRUE,eps=FALSE}


\mode<presentation>
{
  % \usetheme{default}
  \usetheme[width=0cm]{Hannover}
  %\useoutertheme{split}
  \setbeamercovered{transparent}
}

\begin{document}
\begin{frame}
  \titlepage
\end{frame}



% R is well known for its powerful graphics capabilities; however,
% native R graphics is fundamentally limited by its ink-on-paper
% model.  Grid graphics provides enormous flexibility within this
% model, but at the cost of speed.  Part of the flexibility of grid
% arises from notions borrowed from the world of graphical toolkits,
% such as layouts and self-sizing graphical objects.  It is a natural
% next step to explore the direct use of graphical toolkits for
% graphics and other similar tasks in R.


\begin{frame}
  \frametitle{R Graphics}
  \begin{itemize}
  \item Long history
  \item Widely used and tested
  \item Cross-platform, supports multiple output backends \\
    ~~~~(screen, PDF, SVG, pixmap, etc.)
  \item Grid graphics provides enormous flexibility
  \end{itemize}
\end{frame}

\begin{frame}
  \frametitle{Drawbacks of the R graphics model}
  \begin{itemize}
  \item Limited by ink-on-paper model
    \begin{itemize}
    \item Logical graphical primitives not retained at the device level
    \item E.g., a ``+'' plotting character is drawn as two line segments
    \item Things can be added to a plot but not deleted
    \item Interaction difficult
    \end{itemize}
  \item Grid works around this by keeping track of everything it draws
  \item But, 
    \begin{itemize}
    \item Grid is slow
    \item Underlying limitations of the graphics engine remain
    \item E.g., removing a point essentially redraws the whole graph
    \item Doesn't look pretty, because grid is slow!
    \end{itemize}
  \end{itemize}
\end{frame}

\begin{frame}
  \frametitle{Qt}
  \begin{itemize}
  \item Powerful ``Application and UI framework'' written in C++
  \item Widely used and tested
  \item Cross-platform, supports multiple output backends \\
    ~~~~(screen, PDF, SVG, pixmap, etc.)
  \item Qt's Graphics View framework enables flexible graphics
    \begin{itemize}
    \item[] ``...provides a surface for managing and interacting with
      a large number of custom-made 2D graphical items, and a view
      widget for visualizing the items...''
    \end{itemize}
  \item<2>{Can we use Qt to move beyond the limitations of R graphics?} 
  \end{itemize}
\end{frame}

\begin{frame}
  \frametitle{Approach 1}
  \begin{itemize}
  \item Implement an R graphics device using the Graphics View framework
  \item Use Qt's capabilities to enhance functionality
    %%  demo here of graph node etc
  \item But still limited: items in the scene cannot in general be mapped back to the data
  \end{itemize}
\end{frame}

\begin{frame}
  \frametitle{Approach 2}
  \begin{itemize}
  \item A completely independent graphics subsystem 
    \begin{itemize}
    \item Not a novel idea: GGobi, iPlots
    \end{itemize} 
    \item My main interest: implement a Trellis-like system
    \item Interface should be similar for the end-user 
    \item Should make interaction and dynamic manipulation easier
 \end{itemize}
\end{frame}



\begin{frame}
  \frametitle{Why grid?}
  \begin{itemize}
  \item What grid features do we really need?
    \begin{itemize}
    \item Ability to draw arbitrary things in rectangular viewports
    \item Ability to create a layout and put things in it
    \item Control over layout row and column expansion\\ (e.g., panels vs labels)
    \item Basic elements (typically text labels) should know the minimum
      size needed to display themselves
    \item Complex elements (e.g., legends) made up of simpler elements
      placed in a layout should also know their minimum size
      (\textit{``frameGrob''} in grid).
    \end{itemize}
  \item<2>{ All are features that a GUI toolkit excels at }
  \item<2>{ In fact, the design of grid was inspired by GUI toolkits }
  \item<2>{ So, why not just use an actual toolkit? } 
  \end{itemize}
\end{frame}


\begin{frame}
  \frametitle{Mosaiq}
  \begin{itemize}
  \item A high-level lattice-like package
  \item Implemented using the qtpaint API
  \item Makes extensive use of Qt layouts\\ (both Graphics View layouts
    and widget layouts)
  \item Opportunity to clean up API based on lessons from lattice
  \end{itemize}
\end{frame}

\begin{frame}
  \frametitle{API highlights}
  \begin{itemize}
  \item A Trellis-style graph, variables (terms) can be
    \begin{itemize}
    \item Conditioning variables: used to define subsets of data
    \item Panel variables: appropriate subsets used within panel display
    \item E.g., \code{densityplot(\~\ x | a, groups = g, weights = w)}
    \item Finer distinctions exist in the ``Grammar of Graphics''
      worldview, but not in the ``panel function'' philosophy of
      Trellis
    \end{itemize}
  \item The classic Trellis formula API
    \begin{itemize}
    \item Formula defines conditioning and \textit{some} panel
      variables
    \item Others specified using non-standard evaluation paradigm
    \end{itemize}
  \item<2> Problems: 
    \begin{itemize}
    \item Limits code re-use; \\
      special features need to be handled in each high-level function
    \item Difficult to write wrappers/methods; \\
      needs \code{match.call()}, careful handling
    \end{itemize}
  \end{itemize}
\end{frame}

\begin{frame}
  \frametitle{API highlights: plans}
  \begin{itemize}
  \item Separate out specification of ``conditioning variables'' and
    ``panel variables''
  \item Packets defined solely by conditioning variables
    (subscripts); the only thing that differs between panel function
    calls
  \item Panel variables represented as expressions, passed on to
    panel function directly.
  \item Use methods to provide more familiar formula interface
  \item Provide more control over evaluation
    \begin{itemize}
    \item Define a new generic function \\
      ~~~\code{evaluate(e, data, subset, enclos)}
    \item Dispatch (at least) on both \code{e} and \code{data}
    \item Supporting new data types could be as simple as writing a new method
    \item Important for high-throughput Bioinformatics data with complex structures
    \end{itemize}
  \end{itemize}
\end{frame}

\begin{frame}
  \frametitle{Still to do}
  \begin{itemize}
  \item Legends: should be easy, just not done yet
  \item Aspect ratio: same holds
  \item Some clipping issues
  \item PDF outout: currently not vector output for complex widgets
  \item Mathematical annotation (plotmath): not going to happen, but
    should be able to embed R graphics
  \item Interaction model? Will likely involve layers, but needs thought
  \end{itemize}
\end{frame}


\end{document}


\begin{frame}
  \frametitle{}
  \begin{itemize}
  \item 
  \end{itemize}
\end{frame}

\begin{frame}
  \frametitle{}
  \begin{itemize}
  \item 
  \end{itemize}
\end{frame}

\begin{frame}
  \frametitle{}
  \begin{itemize}
  \item 
  \end{itemize}
\end{frame}

